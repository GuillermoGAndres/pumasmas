\documentclass[10pt]{beamer}
\usepackage[utf8x]{inputenc}
\usepackage{hyperref}
\usepackage{fontawesome}
\usepackage{graphicx}
\usepackage[english,ngerman]{babel}
\usepackage{hyperref}

% ------------------------------------------------------------------------------
% Use the beautiful metropolis beamer template
% ------------------------------------------------------------------------------
\usepackage[T1]{fontenc}
\usepackage{fontawesome}
\usepackage{FiraSans} 
\mode<presentation>
{
  \usetheme[progressbar=foot,numbering=fraction,background=light]{metropolis} 
  \usecolortheme{default} % or try albatross, beaver, crane, ...
  \usefonttheme{default}  % or try serif, structurebold, ...
  \setbeamertemplate{navigation symbols}{}
  \setbeamertemplate{caption}[numbered]
  %\setbeamertemplate{frame footer}{My custom footer}
} 

% ------------------------------------------------------------------------------
% beamer doesn't have texttt defined, but I usually want it anyway
% ------------------------------------------------------------------------------
%% \let\textttorig\texttt
%% \renewcommand<>{\texttt}[1]{%
%%   \only#2{\textttorig{#1}}%
%% }

% ------------------------------------------------------------------------------
% minted
% ------------------------------------------------------------------------------
\usepackage{minted}

\usepackage{etoolbox}


% ------------------------------------------------------------------------------
% tcolorbox / tcblisting
% ------------------------------------------------------------------------------
\usepackage{xcolor}
\definecolor{codecolor}{HTML}{FFC300}

\usepackage{tcolorbox}
\tcbuselibrary{most,listingsutf8,minted}

\tcbset{tcbox width=auto,left=1mm,top=1mm,bottom=1mm,
right=1mm,boxsep=1mm,middle=1pt}

\newtcblisting{myr}[1]{colback=codecolor!5,colframe=codecolor!80!black,listing only, 
minted options={numbers=left, style=tcblatex,fontsize=\tiny,breaklines,autogobble,linenos,numbersep=3mm},
left=5mm,enhanced,
title=#1, fonttitle=\bfseries,
listing engine=minted,minted language=r}


% ------------------------------------------------------------------------------
% Listings
% ------------------------------------------------------------------------------
\definecolor{mygreen}{HTML}{37980D}
\definecolor{myblue}{HTML}{0D089F}
\definecolor{myred}{HTML}{98290D}

\usepackage{listings}

% the following is optional to configure custom highlighting
\lstdefinelanguage{XML}
{
  morestring=[b]",
  morecomment=[s]{<!--}{-->},
  morestring=[s]{>}{<},
  morekeywords={ref,xmlns,version,type,canonicalRef,metr,real,target}% list your attributes here
}

\lstdefinestyle{myxml}{
language=XML,
showspaces=false,
showtabs=false,
basicstyle=\ttfamily,
columns=fullflexible,
breaklines=true,
showstringspaces=false,
breakatwhitespace=true,
escapeinside={(*@}{@*)},
basicstyle=\color{mygreen}\ttfamily,%\footnotesize,
stringstyle=\color{myred},
commentstyle=\color{myblue}\upshape,
keywordstyle=\color{myblue}\bfseries,
}


% ------------------------------------------------------------------------------
% The Document
% ------------------------------------------------------------------------------
\title{Vectores en C++}
\author{Club de Programacion Competitiva Pu++}
\date{Abril del 2020}

\begin{document}

\maketitle

\begin{frame}[fragile,allowframebreaks]{}
  Los \textbf{vectores} en c++ son arreglos de \textbf{tamaño dinámico}.
  
  Funcionan de manera muy parecida a los arreglos, tenemos acceso a alguno de sus elementos en tiempo constante (es decir la 'velocidad' de acceso no depende del numero de elementos en el vector) y también podemos tener vectores de distintos tipos de dato cono int, char, string, etc.

  \framebreak

  \textbf{¿Dinámico?}
  
  Así es, no te tienes que preocupar por el tamaño de tu arreglo (vector mejor dicho), c++ se ocupa de eso.
  Para saber el tamaño de un vector ocupanos \textit{size()}.
  
    \begin{minted}{c}
        vector<char> v;  // definiendo el vector v
        v.push_back('n');  // insertando al final
        v.push_back('o');  // insertando al final
        v.push_back('b');  // insertando al final
        
        cout << v.size() << endl;  // 3
  \end{minted}

  \begin{center}
    \begin{tabular}{ |c|c|c| } 
      \hline
      n & o & b  \\
      \hline
    \end{tabular}
  \end{center}


  \framebreak

    Igual que en los arreglos, accedemos a los elementos de acuerdo a su \textbf{índice}. Comenzando por el cero por supuesto.
    
    \begin{minted}{c}
        vector<char> v = {'b', 'o', 'c', 'a',
            'j', 'u', 'n', 'i', 'o', 'r', 's'};
        
        cout << v[0] << endl; // b
        cout << v[2] << endl; // c
        cout << v[0] << v[1] << v[3] << endl // boa

  \end{minted}

  \framebreak

    \textbf{¿Podemos cambiar un elemento?}
    
    Claro. Continuando con nuestro ejemplo anterior:
    \begin{minted}{c}
    
        v[2] = 'l';
        
        for (int i=0; i<v.size(); i++) {  // notar v.size()
            cout << v[i];
        }
        cout << endl;
        
        // se imprime:
        // bolajuniors
    \end{minted}

  \framebreak

  \textbf{Uff, joya pero ¿se puede quitar un elemento?}
    
    Sí. Claro quitemos el \textbf{último} elemento.
    \begin{minted}{c}
        // quitando el elemento con indice n - 1
        v.pop_back();
        
        for (int i=0; i<v.size(); i++) {
            cout << v[i];
        }
        cout << endl;
        
        // se imprime:
        // bolajunior
    \end{minted}
    Esa función solo quita el ultimo elemento, si se desea quitar un elemento distinto a este, se ocupa \textit{erase} pero eso no se necesitará por ahora.
  \framebreak

  \textbf{¿Vectores de vectores?}
  
  En efecto, se pueden declarar matrices con vectores.
      \begin{minted}{c}
        // definiendo el tamaño desde su creación
        // un vector de vectores de tamaño 2
        // eso no define el tamaño de sus vectores 'hijo'
        vector<vector<int>> mat(2);
        for (int i=0; i<2; i++) {
            mat[i].push_back(8+i);
            mat[i].push_back(3+i);
        }
    \end{minted}
    
    La matriz queda:
    \begin{center}
    \begin{tabular}{ |c|c| }
      \hline
        8 & 3 \\
      \hline
        9 & 4 \\
      \hline
    \end{tabular}
  \end{center}

  \framebreak
    \textbf{Arreglo de vectores}
    
    Muy útiles.
    
    \begin{minted}{c}
        const int MAXN = 1e3;  // definiendo una constante
        // arreglo de vectores de int
        vector<int> g[MXN];  // foreshadowing de gráficas
        
        // agregandole un 20 al vector en la posición 2
        g[2].push_back(20);
        
        cout << g[2].back() << endl; // 20
    \end{minted}
    
    La función \textit{back()} regresa el valor del ultimo elemento en el vector.
  \framebreak
  
  \textbf{¿V-vectores de arreglos?}
  
  Jaja sí owo... \textbf{Pero pasemos a otra cosa.} Sobrevivirás sin vectores de arreglos.
  
      \begin{figure}
          \centering
          \includegraphics[width=120px]{a.jpg}
      \end{figure}
  
  \framebreak

  \textbf{Valor default}
  
  El valor default de un vector de enteros es cero no importando dónde se defina.
  Si se desea llenar de un valor distinto se puede:
  
    \begin{minted}{c}
        vector<int> v(3);
    \end{minted}
        
    \begin{center}
    \begin{tabular}{ |c|c|c| } 
      \hline
      0 & 0 & 0  \\
      \hline
    \end{tabular}
  \end{center}
    
    \begin{minted}{c}
        // Inicializando al vector con 1's
        vector<int> v(3, 1);
    \end{minted}
    
    \begin{center}
    \begin{tabular}{ |c|c|c| } 
      \hline
      1 & 1 & 1  \\
      \hline
    \end{tabular}
  \end{center}
  
  \framebreak
  \textbf{Ordenando}
      \begin{minted}{c}
        vector<int> v = {3, 9, 0};
        sort(v.begin(), v.end());
    \end{minted}
    
    \begin{center}
    \begin{tabular}{ |c|c|c| } 
      \hline
      0 & 3 & 9  \\
      \hline
    \end{tabular}
  \end{center}
    No te apures por \textit{begin()} ni por \textit{end()}, solo recuerda que eso necesitas para referenciar \textbf{todo} el vector en alguna función que te pida un rango.
  
  \framebreak
  
  \textbf{Limpiando}
  
  A veces necesitamos limpiar el contenido de un vector para volver a llenarlo, como cuando un problema tiene muchos casos independientes. 
      \begin{minted}{c}
        vector<int> v;
        for (int t=1; t<=T; t++) {
            // usamos el vector
            // ...
            
            // lo limpiamos
            v.clear();
        }
    \end{minted}
    
    Toma en cuenta que el tamaño del vector regresa a ser cero.

  \framebreak
  
  \textbf{Más a fondo}
  
  Documentación de \textit{vector}: \url{https://en.cppreference.com/w/cpp/container/vector}
 
\end{frame}


\end{document}

